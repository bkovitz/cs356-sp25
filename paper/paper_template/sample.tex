\documentclass{article}

\usepackage[
  paperheight=8.5in,
  paperwidth=5.5in,
  left=10mm,
  right=10mm,
  top=20mm,
  bottom=20mm]{geometry}
\usepackage[utf8]{inputenc}

\usepackage{graphicx}
\usepackage{wrapfig}
\usepackage[bottom]{footmisc}
\usepackage{listings}
\usepackage{enumitem}

\usepackage{wrapfig}
\usepackage{ragged2e}

\usepackage{array}
\usepackage[table]{xcolor}
\usepackage{multirow}
\usepackage{booktabs}
\usepackage{hhline}
\definecolor{palegreen}{rgb}{0.6,0.98,0.6}

\usepackage{amsmath}
\usepackage{amssymb}
\usepackage{multicol}
\usepackage{lipsum}
\usepackage{hyphenat}
\PassOptionsToPackage{hyphens}{url}
\usepackage{url}

\usepackage{rotating}

%\usepackage{xeCJK}

%% support use of straight quotes in code listings
\usepackage[T1]{fontenc}
\usepackage{textcomp}
\usepackage{listings}
\lstset{upquote=true}

%% for shrinking space between lines
\usepackage{setspace}

\newcommand*{\affaddr}[1]{#1} % No op here. Customize it for different styles.
\newcommand*{\affmark}[1][*]{\textsuperscript{#1}}
\newcommand*{\email}[1]{\small{\texttt{#1}}}
\newcommand{\tarot}{\textsc{Tarot}}
\renewcommand*\contentsname{\centering Table of Contents}

\renewcommand{\footnoterule}{%
  \kern -3pt
  \hrule width \textwidth height 0.5pt
  \kern 2pt
}

% remove date
\date{}

\usepackage{titlesec}
\titleformat*{\section}{\large\bfseries}
\titleformat*{\subsection}{\normalsize\bfseries}
\titleformat*{\subsubsection}{\normalsize\bfseries}

\usepackage{biblatex}

\addbibresource{sample.bib}

\title{Paper Template and Formatting Requirements\footnote{\protectCopyright \copyright 2025 by the Consortium for Computing Sciences in Colleges.
Permission to copy without fee all or part of this material is granted provided
that the copies are not made or distributed for direct commercial advantage,
the CCSC copyright notice and the title of the publication and its date appear,
and notice is given that copying is by permission of the Consortium for
Computing Sciences in Colleges.  To copy otherwise, or to republish, requires
a fee and/or specific permission.
}
}

\author{
Baochuan Lu\affmark[1] and John Meinke\affmark[2]\\
\affmark[1]Computer and Information Sciences\\
Southwest Baptist University\\
Bolivar, MO 65613\\
\email{blu@sbuniv.edu}\\
\affmark[2]Computer Science Department\\
Another University\\
Our Town, TX 00000\\
\email{jmeinke@univ.edu}\\
}

\begin{document}
\maketitle

\begin{abstract}
This document describes manuscript formatting requirements for CCSC conferences.
Authors can also use this document as a template to format their papers.
\end{abstract}

\section{Length}
Prepare the paper for written understanding with a length of at most ten
(\textbf{10}) single-spaced pages including tables, figures, and a list of references or
bibliography.

\section{Style}
Write clearly and simply in the third person for an audience that is
well-grounded in computing, but who may have limited exposure or knowledge about
the specific topic of your paper. Define any technical terms deemed to require
clarification when they are introduced.

\section{Title and Author Information}
Please follow the example in this paper to enter title and author information.
The title should use title-casing. Please use \texttt{https://titlecaseconverter.com}
with ``AP'' style to reformat your title.

Use the following code in the ``author block'' if there is one author:

\lstset
{
    language=[LaTeX]TeX,
    breaklines=true,
    basicstyle=\tt\scriptsize,
    keywordstyle=\color{blue},
    identifierstyle=\color{magenta},
}

\begin{lstlisting}
  \author{
    John Meinke\\
    Computer Science Department\\
    Another University\\
    Our Town, TX 00000\\
    \email{jmeinke@univ.edu}
  }
\end{lstlisting}

Use the following code in the ``author block'' if there are multiple authors from different department of the same institute:

\begin{lstlisting}
  \author{
    Baochuan Lu\affmark[1], John Meinke\affmark[2]\\
    \affmark[1]Computer and Information Sciences\\
    \affmark[2]Physics Department\\
    Another University\\
    Our Town, TX 00000\\
    \email{\{blu,jmeinke\}@univ.edu}
  }
\end{lstlisting}

Use the following code in the ``author block'' if there are multiple authors from different institutes:
\begin{lstlisting}
  \author{
    Baochuan Lu\affmark[1], Author A\affmark[1], John Meinke\affmark[2], Author B\affmark[2]\\
    \affmark[1]Computer and Information Sciences\\
    Southwest Baptist University\\
    Bolivar, MO 65613\\
    \email{\{blu,author\}@sbuniv.edu}\\
    \affmark[2]Computer Science Department\\
    Another University\\
    Our Town, TX 00000\\
    \email{\{jmeinke,author\}@univ.edu}
  }
\end{lstlisting}

\section{Body of the Manuscript}
The text may be organized into sections and subsections. Please use
\verb+\section+ and  \verb+\subsection+ commands to define them as shown in the
Latex source file of this document. Latex will apply section formatting rules.

\subsection{Abstract}
Provide a one-paragraph brief overview of the paper in both the manuscript for
review and in the final manuscript for publication.

\subsection{Citation}
Appropriately cite all references to other published works included in the paper.
\texttt{biblatex} is used to create a list of references or bibliography as the
last section in the paper. Here are citation examples for a
book\cite{latexcompanion}, a journal paper\cite{einstein}, a
website\cite{knuthwebsite}, and a conference proceeding paper\cite{maurer}.
Please check out the source code of this document for details.

\subsection{Double Quote}
The proper way to typeset double quote is to use two backticks or grave accents
(\`{}) on the left and two single quotes (\'{}) on the right, e.g.
\`{}\`{}Hello!\'{}\'{} for ``Hello!''.

\subsection{Lists}
Lists are easy to create in  \LaTeX\ whether they are ordered, unordered, or
nested as shown in the following example.

\begin{itemize}[noitemsep]
  \item The individual entries are indicated with a black dot, a so-called bullet.
  \item The text in the entries may be of any length.
\end{itemize}

\begin{enumerate}[noitemsep]
  \item The labels consist of sequential numbers.
  \item The numbers start at 1 with every call to the enumerate environment.
\end{enumerate}

\begin{enumerate}[noitemsep]
   \item The labels consists of sequential numbers.
   \begin{itemize}[noitemsep]
     \item The individual entries are indicated with a black dot, a so-called bullet.
     \item The text in the entries may be of any length.
   \end{itemize}
   \item The numbers start at 1 with every call to the enumerate environment.
\end{enumerate}

\subsection{Math Expressions}
The mass-energy equivalence is described by the famous equation

$$E=mc^2$$

discovered in 1905 by Albert Einstein.
In natural units ($c$ = 1), the formula expresses the identity

\begin{equation}
E=m
\end{equation}

\subsection{Tables and Figures}
Include all tables and figures within the body of the text. (Provide as separate
files in the original format any figures so that if there are problems with the
figures coming into the final manuscript there are alternatives available to the editors.)

Here is an example Table \ref{table:nonlin}. Note that each table or figure \textbf{must} have
a title as our style requirement.

\begin{table}[ht]
\caption{Nonlinear Model Results} % title of Table
\label{table:nonlin} % is used to refer this table in the text
\centering % used for centering table
\begin{tabular}{c c c c} % centered columns (4 columns)
\hline\hline %inserts double horizontal lines
Case & Method\#1 & Method\#2 & Method\#3 \\ [0.5ex] % inserts table
%heading
\hline % inserts single horizontal line
1 & 50 & 837 & 970 \\ % inserting body of the table
2 & 47 & 877 & 230 \\
3 & 31 & 25 & 415 \\
4 & 35 & 144 & 2356 \\
5 & 45 & 300 & 556 \\ [1ex] % [1ex] adds vertical space
\hline %inserts single line
\end{tabular}
\end{table}

The trim size of the journal is half-letter size, so please be aware that your
tables or figures may be hard to read if the fonts are too small as shown in
this Figure \ref{figure:big_table}.
\begin{figure}[htbp]
\centering
\includegraphics[width=\textwidth]{big_table.png}
\caption{Fonts are too small.}
\label{figure:big_table}
\end{figure}

In this case, you can either divide the table into multiple parts, or put the
table/figure sideways on its own page as shown in Figure \ref{figure:big_table1}.
\begin{sidewaysfigure}[htbp]
  \includegraphics[width=\textwidth]{big_table.png}
  \caption{A big table turned sideways on its own page.}
  \label{figure:big_table1}
\end{sidewaysfigure}

The code for inserting figures sideways is as follows:

\vbox{
\begin{lstlisting}[language=TeX, basicstyle=\footnotesize,
caption=Code to Insert Figures Sideways]
\begin{sidewaysfigure}[htbp]
  \includegraphics[width=\textwidth]{big_table.png}
  \caption{A big table turned sideways on its own page.}
  \label{figure:big_table1}
\end{sidewaysfigure}
\end{lstlisting}
}
Here is a live example: \url{https://www.overleaf.com/read/fzcmnctdppgz}

\subsection{Use of Color}
Even though our digital journal is in color, our print journal is in black and white.
You may choose to use other colors in your paper, but please be aware that the colors
you choose may not be distinguishable after they are turned into gray-scale.

\subsection{Reference List}
The \verb+\printbibliography+ command prints a list of references for you.
Please use \texttt{sample.bib} as an example to create your bibliography entries.

If you can find a reference on \url{https://scholar.google.com/} you can get a
``BibTex'' export by clicking on the quotation mark symbol, which is a lot easier
than entering the information manually.

\subsection{Code Listings}
Commands from \texttt{listings} package allow you to display code easily with
customizable coloring and styling rules. Here is an example.
Please check out the source code of this document for details.

\begin{lstlisting}[language=Python,  caption=Python example]
x = 42
epsilon = 0.01
step = epsilon**2
num_guesses = 0
ans = 0.0
while abs(ans**2-x) > epsilon and ans < x:
    ans = ans + step
    num_guesses += 1
if abs(ans**2-x) <= epsilon:
    print(str(ans) +
    ' is close to the square root of ' +
    str(x))
else:
    print('Failed to find square root of ' + str(x))
print("The number of guesses is " + str(num_guesses))
\end{lstlisting}

\section{Manuscript Submission}
The following materials will need to be submitted:
\begin{enumerate}[noitemsep]
  \item The final manuscript in Latex.
  \item Copyright release. It is essential that we receive the copyright
  release form. By signing this form you are acknowledging that the manuscript
  has not been printed in another venue, plus you are retaining your rights for
  use of the manuscript. Read the copyright release. The Consortium will not
  prohibit you from using the manuscript, but will ask that you credit any reuse
  to the Consortium as the original source of publication. If you misplace the
  copyright form, a generic copyright form can be found through the Copyright
  Release Form\cite{copyright}.\\
  The Consortium encourages multiple presentations of tutorials and workshops.
  If you are presenting a tutorial or workshop you may retain the copyright,
  but we must have that documented. Keep in mind that your manuscript is limited
  to two pages total. However, you must still submit a copyright form.\\
  Please note that it is critical that you obtain permission to use third party
  material. If you use diagrams and such that are attributable to a third party
  you must obtain formal permission to reprint such items, and must so indicate
  in the copyright release as well as submit such permission.
  \item Registration for the conference, along with the appropriate registration
  fee. We have found that there are some folks in need of publication for
  promotion and tenure purposes, and then don’t want to present the paper.
  A major plus of the Consortium conferences is the presentation of the papers,
  and you must plan on attending. If you do not present the paper at the
  conference the paper will be removed from the ACM Digital Library.
  \item A pdf version of your manuscript is most helpful. If there are problems
  with special characters or special formatting this provides the editors with
  what you expected your final manuscript to look like. Providing a pdf version
  or a hard copy helps significantly in envisioning what the author expected
  the final product to look like.
  \item Electronic copies of any graphics in a standard format (bitmap, jpeg, tiff).
\end{enumerate}

\section{Additional Information}
Please feel free to email \verb+ccsc-editors@googlegroups.com+ for questions.
This document is modified from the CCSC manuscript formatting
document\cite{meinke} created by John Meinke.

\medskip

\printbibliography

\end{document}
