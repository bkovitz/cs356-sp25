\documentclass{article}

\input{preamble}

\usepackage{verbatim}

\addbibresource{mobdesign.bib}

\title{Mob Programming in a Software Design Course}

\author{
%Ben Kovitz \\
%Department of Computer Science \\
%California State Polytechnic University, Humboldt \\
%Arcata, CA 95521 \\
%\email{blk14@humboldt.edu}\\
Anonymous Author \\
Department of Computer Science \\
Anonymous University \\
Anonymity City \\
\email{anon@example.com}
}

\begin{document}
\maketitle

\begin{abstract}
Case study of a Software Design course taught using mob programming. Mob
programming (also ``team programming'', ``ensemble programming'') is
collaborative programming in which the whole team---in this case, the
whole class---writes code together at the same time, like pair programming
but including everyone. The course combined daily collaborative design and
coding with intensive daily readings and oral presentations. Students
gained most strongly in skills that are normally hard to teach in
colleges: egoless cooperation and ``tacit knowledge'' of software design.
\end{abstract}

\section{Introduction}
I taught a course on software design in the Spring~2025 semester at
California State Polytechnic University, Humboldt (course number CS~356),
predominantly by a new pedagogy: mob programming. Mob programming,
also called team programming or ensemble programming, is ``a software
development approach where the whole team works together on the same thing,
in the same space, and at the same computer.'' \cite{zuill2022software}
Mob programming is like pair programming, but not limited to two programmers
at one computer.

Several reasons suggested mob programming as an effective way to teach
software design:

\begin{enumerate}
  \item Experience from industry suggests that pair programming and mob
  programming result in knowledge spreading rapidly throughout the team.
  %@@CITE
  So why not exploit this happy property of mob programming as a class's
  principal method of learning?

  % TODO Delete 1st sentence?
  \item Many industry complaints about skills lacking in recent college
  graduates revolve around ``habitability''---the aspects of design that
  go beyond correctness and relate to programmers' ability to work with
  code over many years, such as readability, maintainability, testability,
  and debuggability \cite{Gabriel1996}. The usual college-course format
  of lectures, homework assignments, and exams often gives students
  experience only with writing small, stand-alone pieces of code that just
  barely work or partly work (enough to get partial credit) and that no one
  looks at ever again. Could mob programming on one project, building it for
  most of one semester by piecemeal growth, give students valuable experience
  working on each other's code, seeing for themselves how to grow a 
  software design that's habitable for others?

  \item Software engineering is a complex and collaborative art.
  Engineers must constantly learn and constantly adjust to unexpected problems
  and to other people on the development team. Much of the skill is difficult
  or impossible to express precisely in verbal representations and procedures,
  and can be acquired only through real practice. This makes it a good fit for
  the research by anthropologists Jean Lave and Etienne Wenger on learning in
  ``communities of practice''. \cite{wenger2015introduction} Could mob
  programming create an artificial community of practice for one semester?

  \item Only seven students were enrolled in the class. CS~356 is a
  junior-level course, and these students were among the first cohorts in the
  university's recently begun Software Engineering degree program. So,
  this semester provided a good opportunity to experiment with mob
  programming.
\end{enumerate}

The rest of this article describes how the course went: its structure,
problems that arose, solutions that we found or didn't find, and a final
assessment.
%Where appropriate, I'll follow a loose version of Portland Form,
%briefly describing each problem and following it with a paragraph marked
%``Solution:'' to indicate the solution chosen.

\section{The Semester}

\begin{comment}
Solution: I took the union of all theories of design, and announced on
the first day that the entire field of software design is a matter of
opinion (including this very statement). % TODO throw this out?

% readings & presentations

% time for retrospection
\end{comment}

\subsection{Semester Overview}

The semester plan was as follows. (The complete syllabus is available at
\cite{kovitz2025cs356syllabus}.)

\begin{itemize}
\item Each class session was a lab---no lectures, except for an introductory
lecture on the first day and occasional improvised mini-lectures as needed.
Sessions were one hour and 50~minutes long, twice a week.

\item Readings were assigned at the end of each class session, and students
had to give brief presentations on the readings at the start of the next
session. These presentations usually took up the first 20 to 30 minutes of
class time. The readings came only from sources influential among
professionals, such as books by Martin Fowler and Scott Meyers, as well as
famous blog posts and journal articles, not from textbooks.

\item Students were required to keep notebooks in which they recorded the
ideas or techniques they learned in each class session, and to periodically hand
these in during the semester.

\item The first few class sessions: design exercises, mostly omitting
implementation, to teach UML and design as something distinct from code.

\item Most of the semester (about 10~weeks) was spent designing and
implementing a single piece of software in C++, collaboratively by mob
programming together in class.

\item Last two weeks: we returned to design exercises without implementation,
now introducing distributed system design (load balancers, microservices,
etc.).

\item Last day of class: each student gave a presentation on what they learned
during the semester.
\end{itemize}

\subsection{Initial Design Exercises}

The first design exercise was borrowed from a job-interview question,
involving designing a state-transition diagram for a controller in a common
home appliance (details omitted so I can reuse it). The students did this in
about an hour of class time, after which I corrected errors and pointed out
that they had worked out a mathematically complete description of the software
without writing a line of code.

Then we moved on to implementing the finite state machine in C++. Had this
been an ordinary class, I would have written the code at a projector,
explaining my thoughts as I went. Instead, we had our first experience with
mob programming, which proved much more educational. We set up our mob
``rotation'' as follows:

\begin{itemize}
\item One person, called the ``driver'', sits at the keyboard of the computer
hooked up to a projector.

\item One person, called the ``navigator'', tells the driver what to do next.

\item Everyone else (``the mob'') watches, checks the code for errors, sees
what ideas come to mind, and does anything else they think might be helpful,
such as adding to-do items to the whiteboard or searching the Internet for
documentation.

\item We followed Llewelyn Falco's ``strong-style
pairing'' \cite{falco2014strong}, which has this rule: ``For an idea to go from
your head into the computer, it must go through someone else's hands.'' The
driver does not enter his or her own ideas into the code. The driver's
job is to follow the navigator's instructions. The navigator tries to state
each instruction at the highest level of abstraction that the driver can
implement. That may take the form of ``Now write a test for the null case''
or, if necessary, specific keystrokes to help the driver operate an unfamiliar
text editor.

% TODO Explain the importance of strong-style pairing for verbalizing and
% cooperation

\item We set a timer so that every 5~minutes, we rotated roles. Someone from
the mob becomes the driver, the driver becomes the navigator, and the
navigator returns to the mob.

\item All the students' names were written in sequence (``the rotation'')
on the whiteboard, so we always knew who was to drive next. I placed myself
eighth (last) in the rotation so that every student would have a chance to
navigate before I spoke up.
\end{itemize}

We drew the state-transition diagram on the whiteboard, and then the students
proceeded to do exactly what I think you should not do: they started putting
together tricky nested loops with tricky conditions inside---essentially
neglecting the state-transition diagram and trying to reproduce the desired
output cobbling together familiar programming constructs in the usual ways.
%We
%quickly evolved a style of mobbing where people other than the navigator and
%driver feel free to call out ideas. But I held silent.

At one point, the navigator suggested ``reversing'' a variable for a sensor's
input. The driver didn't know how to do that in C++. The navigator said, ``Put
an exclamation point'' (the Boolean negation operator). The driver was
surprised by how easy that was, got a brief explanation, and then went on to
the next thing. Similar small points came up during mobbing nearly every day.
Students came to the class with widely varying prior experience. Some might be
lost in a lecture that assumes knowledge they don't have and get further
behind as the semester continues.  Mob programming effortlessly paused to fill
in just the bit of knowledge that a student needed, right when he or she
needed it.

%While the course was intended to teach
%``software design'', a rather high-level concept, it also taught a great many
%small details

Eventually the students hit on the idea of a variable for the state of the
state machine. From the mob, I suggested a \texttt{switch} statement,
but they still had no systematic approach and were writing
extremely bug-prone code. Finally, as the navigator, I had the driver make a
\texttt{case} for the \texttt{switch} statement that fully handled the initial
state.  The students had ``aha!'' moments and, as they rotated into the
navigator role one by one, worked out how to systematically cover all of the
infinite set of possible input sequences. On our first attempt to compile the
code, it was correct and passed all tests (which the students were now able to
design systematically, too). The students had their first experience with
\emph{design} distinct from code, and with a systematic mapping of design
elements to program elements.

Some students found the initial mob programming stressful. They'd written
almost all of their previous code solo. Now they had to write code in front of
a whole class---while thinking it up, unsure of their approach, gaps in their
knowledge and skills on display. ``Am I doing it right?'' some asked me after
class, unsure, since I gave them no rubric for ``correct'' mob programming.

A small problem with the physical environment became apparent: the tables in
the lab room were far apart and perpendicular to the projection screen, so it
was awkward to look at the screen, and background noise made it hard to be
heard across the room. We eventually found a quiet room where the seats faced
the projection screen and a new driver could swap into place in a few seconds.
Mobbing can be done hovering around a laptop, but these conveniences help.

After that, we moved on to designing an ATM machine,
%using Russell Bjork's ATM Simulation site \cite{bjork_atm_simulation} and
following the method of
making classes from nouns in the requirements and assigning responsibilities
based on relationships, as explained in \cite{wirfs1990designing}. We liked
the mob-rotation format so much, we used it for this UML-only, whiteboard-only
design process. There was no navigator role, only a driver role that rotated
every 10~minutes. The driver wrote on the whiteboard while everyone else
suggested items to add or change. Since I was the only one who knew the
process, I facilitated.

\subsection{The Project}

Before the semester, I had asked around campus for software that people needed
written and even received some suggestions on LinkedIn. In class, we rated all
the options at the whiteboard for feasibility, opportunity to learn design
patterns, and suitability for C++, and chose to make a diagram editor. In this
as in all our group decision-making, we followed a method similar to ``dot
voting'' as used at Norman Nielsen Group \cite{budiu2024dotvoting} and
commonly done in mob-programming circles. We kept a clear boundary between
exploration (``divergent thinking'') and evaluation or filtering of ideas
(``convergent thinking''), doing first one and then the other.

We spent one class session brainstorming for features. This gave the
students experience in collaborative software design at the level of
requirements or product definition. The first idea was ``little boxes in a
canvas''. Exploration continued through ``TiKZ editor'' and, most ambitiously,
``the VSCode of diagram editors, allowing plug-ins for electrical simulation
of schematic diagrams or anything else''. We pared that down to ``enough for
Ben to draw graphs for homework problems in graph theory, with a
\emph{vim}-like user interface.''

Next we had to make a crucial design decision: which GUI library to use?
I assigned readings on two popular and relatively simple libraries, FLTK and
SDL, intending that the students would choose one of those. One student
independently researched Qt, a much more complex library---which I had
intended to avoid. After presentations on all of them and another group
discussion, we chose Qt, attracted by its ability to make a sophisticated
GUI in C++ very quickly and run it on both desktop and mobile platforms,
deeming this to outweigh the risks of greater complexity and time to learn.

I assigned readings in the Qt documentation and we got back to mob
programming.  Qt did turn out to be problematic. We spent a lot of time in and
out of class getting the Qt libraries to install and work with VSCode on the
students' various computers and operating systems.  This was chaotic and
frustrating---but so is real-life software development, often for reasons just
like this. Qt did enable us to quickly and easily try out out radical new
user-interface ideas, something we likely could not have done with the other
libraries.

We found that mob programming can work at amazing speed---even as it feels
like you're working slowly, doing just a tiny bit before passing the baton to
the next person.  One day, for example, after we implemented snap-to-grid as
the user moved an object, we had five minutes left, and our next item was to
make the grid visible---too little time, I thought. But the next
navigator/driver team dove in, and with a little help from the mob got Qt
displaying a visible grid before class was done.

We experimented with the rotation timer and found that we liked a 7-minute
rotation best. That was long enough to let the navigator and driver settle in
and still allowed each student to get two rounds in during the class period.

One day, a student mentioned that he had no idea how to implement Undo.
I assigned readings on the Command and Memento patterns, and the next day,
after oral presentations, the class weighed them at the whiteboard and chose
Memento because it seemed simpler to implement: just store a
vector of snapshots of the current state of the diagram. By the next class
session, we found the flaw: coordinating ownership of Qt's diagram objects was
unmanageable. It emerged that the students had never fully grasped that in
C++, unlike in Java or Python, you distinguish between holding a reference to
an object and holding the object---and the pitfalls of having to bear this
responsibility. We deleted the code and reimplemented using the Command
pattern---and found not only that it worked correctly, it was actually simpler.

This experience---completely unplanned---implicitly taught a fact about
software design that is easy to say but hard to believe until you've
experienced it: it's often difficult to tell whether a design idea will be
good or bad until after you've implemented it. In hindsight, it's easy to say
that we ``should'' have chosen Command on the first day, but of course that's
only in hindsight. 

\subsection{Testing the edges of mob programming}

The Memento/Command revision prompted us to stray from strict mob
programming and stop the timer for about 45 minutes---the time needed to
explain and reach consensus about the problem, including understanding that Qt
was written long before C++11's \texttt{unique\_ptr} and \texttt{shared\_ptr}
and follows different conventions for ownership of dynamic memory.

The navigator role turned out to induce the most stress. As
driver, you're sitting at the keyboard typing code in front of everyone, but
as navigator, you're standing in front of the everyone and you bear the
responsibility of directing the work. Sometimes a student didn't have ideas
yet, or his or her ideas weren't yet clear enough to tell someone else how to
implement them. Our mobbers tended to call out ideas while the navigator was
still gathering his or her thoughts.
Our solution: When you're the navigator in this position, ask, ``May I have
a moment of silence, please?''

We found that some design ideas required more thought, or a different
kind of thought, than happens during mob programming. Sometimes, just getting
an idea across takes longer than one rotation. So, on some days, I assigned
to one person not a reading, but a design problem and a first implementation.
In the next class session, that person gave an oral presentation on what they
made, and then the class as a whole reimplemented it from scratch. This
produced a result that was often simpler and was of course understood by the
whole class, since every student literally had their hands in writing it.

\begin{comment}
``common knowledge''

The day I improvised a lecture on agile vs. waterfall. Previous exposure had
given some students the impression that agile is hopelessly high-risk
because it has no discipline and other
students the impression that waterfall is hopelessly high-risk because it's
inflexible and assumes perfect forethought.

Some students had previous experience with Git and some did not.

No textbooks.

Danger: what if a prima donna shows up?

quiet students not asking questions, not having their problems addressed---not
a problem now
\end{comment}

\subsection{AI-Assisted Programming}

The class's AI policy was to use AI mainly for documentation look-up and simple
code-completion with Copilot during the first half of the semester, and during
the second half, to run as wild as we could with AI and see all that it can do.
Right after Spring Break, a student suggested writing a short document
describing all the features we wanted and letting ChatGPT rewrite the entire
diagram editor from scratch.

So, we mob-wrote a user-interface design document in a wiki page and let
ChatGPT rip. Its code did not compile, nor did revising the document to tell
it not to make those errors again fix it.  We manually fixed the problems and
made some rapid progress.

But then the class rebelled! The students found that the fast cycle time was
so fast, they did not understand the code that ChatGPT was producing. They
wanted to understand the programming and design ideas, not merely see them
rushing by. So we reverted to our AI policy from the first half of the
semester, and actually used Copilot more sparingly.

\begin{comment}
\subsection{Other Problems and Solutions}

Here are a few other problems that emerged and the solutions we chose.

\emph{Problem:} The term ``software design'' is highly ambiguous.
Different experts hold radically different conceptions, from
Jack Reeves' view that ``the source code is the
design'' \cite{reeves1992software} to IEEE~1016--2009, which specifies
12~``design viewpoints'', each with its own design entities, design
relationships, and design attributes, to be described in a design language
such as UML \cite{IEEE1016-2009}. The number of aspects or elements of software
design is mind-boggling, and each comes with its own opinionated camps and
controversies: high-level design, low-level design, system design, interface
design (which can mean user-interface design, interaction design,
communication protocols, or class method signatures, depending on who's
talking), database design, architectural design, GoF design patterns, ways of
organizing source code into subroutines, and more. How can one course cover so
many topics, and steer between the Scylla and Charybdis of pushing a
controversial position on the students and letting students wander without
guidance into completely uninformed opinions?

\emph{Solution:} The first readings were on three radically different
conceptions of design, one by Dave Parnas, one by Alan Cooper, and the above
article by Jack Reeves.
\end{comment}

\section{Assessment and Conclusions}

\subsection{Results}

The students reported two main things that they learned from the class:

\begin{enumerate}
\item How to collaborate. The students had all done some pair programming in
earlier classes, but they had never seen collaboration at this intensity. I
had meant for mob programming to give the students broad practical experience
with software design, and it did, but collaboration itself took the
starring role.

\item The parts of software development that are too hard to put into words.
For example, how much forethought should you put into code before writing
it? How quickly does refactoring lead a good design to emerge
``spontaneously'' and under what circumstances does this happen or fail to
happen? How much design is overdesign?
\end{enumerate}

The students reported that they got extraordinary practice in oral
communication, partly from the daily oral presentations and partly from the
discipline of verbalizing every design decision that mob programming imposes.
Mob rotation circumvented the common problem of vocal students dominating a
class and quiet students not asking about their concerns.

The biggest disappointment of the course was that 10~weeks was not enough to
implement a satisfactory ``graph editor for homework assignments''. Ten weeks
sounds like a long time, but meeting twice a week, that amounts to only
about 30~hours of mob programming.  That's less than a week of work at a
regular job, plus the students were learning many unfamiliar
concepts.

Another disappointment was that the students got only a little exposure to
unit tests and test-driven design. The students all made a test or two in
GoogleTest, but we never used it in the diagram editor because unit-testing
a GUI is hard. I would recommend in future mob-programming courses to
implement a nontrivial back-end for a web site with a simple interface, or
simply a program that has only a textual user interface.

The course was in some ways easy to teach and in other ways hard to teach.
The students did most of the lecturing, not me.
%Most days, I showed up and
%mobbed more as a colleague than as a teacher, contributing the same way that I
%would in industry.
However, the course also required me to pay close attention
and adjust continuously and creatively throughout the semester, more than in
other classes. While I had selected a set of readings before the semester, I
had to choose and search for readings before each class to suit whatever the
programming was leading us to next.

The syllabus listed 78~topics of which I hoped to cover ``some large subset''.
I count about 55 of them touched on or explored to varying degrees: the DRY
principle, \emph{git} and version control, refactoring, code smells,
information-hiding,
Model--View--Controller and other design patterns, adding a level of
indirection, writing the calling code first, YAGNI, coupling and cohesion, and
many more. Most of these topics came up naturally, without lectures, as they
arose in the course of writing code and talking about it; some needed
readings.

The course's focus on tacit knowledge gained through experience made designing
the final exam tricky.  Nevertheless there was much explicit knowledge that
could be tested: designing a finite-state machine to solve a problem, making
UML diagrams given requirements for a simple distributed system, articulating a
reason for an opinion about software design. The final exam demonstrated
weakness in the students' understanding: most answers to the ``hard'' problems
had some serious flaw. Perhaps that's to be expected in a course with more
breadth than depth, but I would recommend in future mob-programming courses
that a homework assignment to be done solo should follow each central topic,
even if this means going a day without new readings.

Course evaluations averaged 4.8 on a scale of 0 to 5, the highest I've
received as a teacher.  I am most pleased that the students \emph{know} what
they learned in the course. Their knowledge is imperfect, but they know it
first-hand, not ``because I was told that in college.''

\subsection{Risks and Unknowns}

Three important unknowns are unaddressed by this experience. First, what if a
student is uncooperative? An occasional problem in industry, more common in
people of college age, is the intransigent ``prima donna''---the opinionated
programmer who knows better than anyone else, sees faster than anyone else
what to do, and sees no value in patiently working to build consensus.

Second, what if a student is unable to keep up? Programming is hard,
especially at the nearly professional level reached in this course, and not
all students have the talent or prior knowledge needed to do it. 
Some students hold back and let their teammates do most of the work---a
common problem in group projects, even at the senior level.

In ordinary classes, a few unpleasant or lazy students don't bring down the
class. But in an intensely collaborative class, they could spoil the
experience for all.  I was fortunate to have a class filled completely with
cheerful, dedicated, cooperative students; not all mob-programming classes may
be so lucky.

And third, could the mob-programming approach be brought to a larger class?
A mob of eight is already quite large, requiring each student to wait a long
time between turns as driver; a mob of 30 would be impossible. One idea is to
have multiple mobs of about 4~students, with the teacher and/or students
``floating'' between different mobs, carrying information along and being
brought up to speed upon joining each new mob. Would this require multiple
projectors? If so, the expense might be prohibitive.

\section{Acknowledgements}

Much gratitude to the many people who helped make the course a success:
Shahzad Aslam-Mir for wide-ranging advice on everything from projects to tools
to C++ libraries; %, which silently averted disasters and made things run
%smoothly;
Rebecca Wirfs-Brock for inviting me to look into mob programming and
guidance finding the right balance between giving too much and too little
direction to the students; Woody Zuill for generously making time to answer my
many basic questions about mobbing; Austin Chadwick and Chris Lucian for
generously making time to answer yet more questions about mobbing and letting
me mob remotely at Hunter Industries; Dave Bender for the initial design
exercise;
%Russell Bjork for his ATM simulation;
Sherrene Bogle for creating the
original CS~356 course, making this experiment possible; and most of all to
the students, who dove boldly, cooperatively, and creatively into a wildly
experimental course.

%\medskip

\printbibliography

\end{document}
