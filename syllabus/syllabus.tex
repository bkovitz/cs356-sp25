\documentclass[12pt]{article}
\usepackage{array}
\usepackage{booktabs}
\usepackage{fontspec}
\usepackage[margin=1in]{geometry}
\usepackage{tabularx}
\usepackage{multirow}
\usepackage{hyperref}
\usepackage{colortbl}
\usepackage{nicematrix}
\usepackage{xcolor}
\usepackage{changepage}  % for the \adjustwidth environment
%\setmainfont{Times New Roman}

\newenvironment{col3}
  {\begin{minipage}[t]{0.3\textwidth}
   \fontsize{10}{12}\selectfont
   \raggedright
   \leftskip 1em
   \parindent -1em}
  {\end{minipage}}

% A heading in the list of topics
\newcommand{\hd}[1]{\vspace{\baselineskip}\textbf{#1}\vspace{0.1em}}

\begin{document}

\begin{center}
\textbf{\Large CS 356: Software Design}

\textbf{Cal Poly Humboldt}

\textbf{Department of Computer Science}

\textbf{Syllabus---Spring 2025}
\end{center}

%\vspace{1em}

\begin{center}
\arrayrulecolor[gray]{0.6}
\renewcommand{\arraystretch}{1.5}
%\setlength{\extrarowheight}{6pt} 
\begin{tabularx}{\textwidth}{@{}>{\bfseries\raggedright\arraybackslash}p{0.18\textwidth}p{0.78\textwidth}@{}}
\toprule
Labs & Tuesdays 3:00--4:20 room FH 202 \\
\midrule
Course Credit & 3 units \\
\midrule
Instructor &
\begin{tabular}{@{}p{0.34\linewidth}p{0.24\linewidth}p{0.35\linewidth}@{}}
Ben Kovitz & \textbf{Office} & BSS 344
\end{tabular} \\
\midrule
Email &
\begin{tabular}{@{}p{0.34\linewidth}p{0.24\linewidth}p{0.35\linewidth}@{}}
blk14@humboldt.edu & \textbf{Office Phone} & (707) 826--3492 \\
\end{tabular} \\
\midrule
Office Hours & MW 3:00--4:00 and by appointment \\
\midrule
Texts &
\parbox[t]{\linewidth}{
We will be reading excerpts from the following:

  \vspace{0.5em}
  \begin{minipage}{\linewidth}
  \setlength{\parindent}{1em}
  \setlength{\hangindent}{1em}
  \setlength{\parskip}{0.3em}
  \emph{Design Patterns} by Erich Gamma, Richard Helm, Ralph Johnson, and John
  Vlissides (the ``Gang of Four''). 1994.

  \emph{UML Distilled,} 3rd~ed., by Martin Fowler. 2003.

  \emph{Effective C++,} 3rd~ed., by Scott Meyers. 2005.

  \emph{Effective Modern C++} by Scott Meyers. 2015.

  \emph{Refactoring,} 2nd~ed., by Martin Fowler. 2019.

  \emph{Design Patterns in Modern C++20} by Dmitri Nesteruk. 2022.
  \end{minipage}
  \vspace{0.3em}

All of these books are available free on-line to Humboldt students
at O'Reilly Learning: \url{https://go.oreilly.com/humboldt-state-university}.
All but the last are also available fairly inexpensively in hardcopy as used
books (recommended).
}
\\[3.0ex]
%\midrule
%Course Description & Software Design.
%Learn and practice software development lifecycle: secure design, modeling, architecture, verification and validation. Complete team projects, produce technical documents, and make formal presentations.
%3 credit hours.
%\\
\midrule
Prerequisites & CS~201 (Requirements Engineering); concurrently: CS~325
(Database Design) \\
\midrule
Canvas Link & \url{https://canvas.humboldt.edu/courses/76623} \\
\midrule
Final Exam & Tuesday, May 13, 2025, 3:00--4:50~p.m.\ in room FH~202 \\
\bottomrule
%\end{tabular}
\end{tabularx}
\end{center}

\subsection*{Course Description}

In this course, you will learn how to design software, mainly by designing
software---and implementing it. And you will learn partly by reading and
discussing classic writings about software design.

This course will be all labs, no lectures. Here's how the course will work:

\begin{itemize}
\item The first few class sessions will be design exercises, mostly omitting
implementation in code.

\item After that, we will spend most of the semester developing a large
piece of software---designing \emph{and} implementing. All code will be
written during class, collaboratively, by ``mob programming'' (see below). 

\item There will be assigned readings before every class session. Some will
come from the texts listed above and some will come from articles that I find.
Expect about 10--40 pages of reading a week.

\item Each reading will have one student assigned to give a short presentation
on it---three to five minutes at the start of the next class. We'll have a
brief in-class discussion after each presentation.

\item After each class session, you are required to write some notes
summarizing what you learned that day---one or two paragraphs of complete
sentences, up to a page if you like. I'll read these periodically to see how
the class is going.

\item Use of AI is allowed and encouraged. During the first half of the
semester, we will limit AI to documentation look-up and debugging. In the
second half of the course, we will run wild with AI, using it for everything
we possibly can, seeing what works and what doesn't work.

\item We might spend the last few sessions of the semester on some high-level
design exercises of the sort that are common in job interviews.
\end{itemize}

Along the way, you will learn:

\begin{itemize}
\item How to design and implement software collaboratively

\item How to sketch out a design with UML diagrams

\item Design patterns

\item How to improve a design by refactoring

\item Basics of user-interface design

\item How to use modern C++ effectively (beyond basic object-orientation)
\end{itemize}

This is a highly experimental course. Many things will go wrong.
I have never done mob programming myself, and I've only recently been
experimenting with AI-assisted programming. We will all be learning this
at the same time.
Expect many ``course-corrections'' throughout the semester.

You must show up! This is class is pretty much all in-class activity.
Attendance and active participation during class are mandatory.

We will choose the exact project after we finish the design exercises.
I have asked people around campus for software that they need written.
We might choose one of theirs or we might choose a project of our own.
The constraints are that C++ must be a reasonably good fit as an
implementation language, the project must be complex enough that it
will call for a variety of design patterns, and it must be feasible
to implement within the semester.

\subsection*{Mob Programming}

Mob programming is a way of writing software ``where the whole team works
together on the same thing, at the same time, in the same space, and at
the same computer.'' (\emph{Software Teaming,} 2nd~ed., by Woody Zuill and
Kevin Meadows, 2022, p.~2.)

During mob programming, one person, called ``the driver'', is at the keyboard.
Everyone else should be able to see their screen. The team rotates through
the driver role every 7~minutes (or whatever rotation time we set---we'll
experiment).

Another role is ``the navigator''. The navigator tells the driver what to do
next. The driver functions as an intelligent input device, translating
higher-level ideas into keystrokes and code. A maxim: ``For an idea to go from
your head into the computer, it must go through someone else's hands.''
---Llewellyn Falco.

We will explore roles that the other team members can play. Generally you
should be checking what's going on at the keyboard for errors or better ways
to do it, discussing them with the navigator, and finding other ways to
support the team.

We'll try to make most decisions by consensus, but occasionally there will be
disagreements. When the time required to try two ideas is short, we'll try
them both rather than argue. Sometimes, though, we will need to make a group
decision that not everyone agrees with. At these times, the principle to
follow is ``Disagree and commit''---that is, commit to implementing the team's
idea and making it work as well as you can, even if you disagree with it.

Near the start of the course, all of us will work as one mob so we all learn
the basics. After we know how to do it, we'll likely break frequently into
smaller mobs that work in parallel.

\subsection*{Your Time Outside of Class}

Since all code will be written in class, you won't need to write code as
homework nor will you need to schedule times with other students on team
projects. You should still block out about 2--4 hours of uninterrupted
time each week, though, for the readings, for preparing your presentations,
and for making your after-class notes. During some weeks, we may assign some
research work to help with the project, or you might simply want to explore or
code up an idea on your own initiative. So, allow that in some weeks, you
might need as much as 6~hours of time outside of class.

\pagebreak
\subsection*{Topics}
We will cover some large subset of the following topics. The exact topics will
result from the readings and from the problems and techniques that arise
organically in the course of designing and implementing the software for the
semester project. At the end of the semester, we'll look over this list and
see which topics we covered---and what we learned about design that was not
on the list.

\vspace{-0.5\baselineskip}
\begin{col3}
\hd{OO Analysis}

nouns $\rightarrow$ classes

verbs $\rightarrow$ methods

UML class diagram

UML sequence diagram

UML state diagram

UML use-case diagram

UML activity diagram

design documents

\hd{Deciding What to Do Next}

test-driven design (TDD)

YAGNI

Write the calling code first

Top-down or bottom-up? Start from what you know

Premature optimization is the root of some evil

stubs

\hd{Planning}

agile \& waterfall

the cost-of-change curve

risk-mitigation
%risk-reduction

%continuous integration
build one feature at a time

learning happens during software development

pauses to learn something

spikes


\hd{Ancillary but Important}

requirements / specifications

unit tests

git / version control


\hd{Good and Bad Code}

refactoring

coding standard

clear names > comments

code smells

technical debt
\end{col3}
\hfill
\begin{col3}
\hd{General Concepts \& Maxims}

information-hiding

What is the simplest thing that could possibly work?

Choose an output for every possible input

the DRY principle (aka OOAO)

Design code to be testable
%(small modules easily separated from their context)

Favor immutability

Fail visibly

abstraction layers

leaky abstractions

look-up table

Every problem in computer science can be solved by adding a layer of indirection
…except for the problem of too many levels of indirection

Make representations that have no invalid states

%A table-driven or pure-functional design is sometimes much simpler and less error-prone than an OO design

coupling and cohesion

%encapsulation: keep related things together

design by contract

ripple


\hd{OO Maxims}

aggregation > inheritance

S: single responsibility

O: open/closed principle

L: Liskov substitution principle

I: interface segregation

D: dependency inversion

Law of Demeter

\hd{C++}

resource acquisition is initialization (RAII)

Curiously Recurring Template Pattern (CRTP)

rule of 3 / rule of 5

smart pointers
\end{col3}
\hfill
\begin{col3}

\hd{GUI Design}

user model vs. implementation model

paper prototypes

Principle of Least Amazement

user thinks out loud

%user points at paper prototype
%(Could we get a student or two who are not in the class to serve as
%users during prototyping?)


\hd{Design Patterns}

Wrapper (Adapter)

Factory, factory method

Observer

Composite

Decorator

Model–View–Controller

Interpreter

State

Strategy

Iterator

Template methods, template classes

Visitor

value objects

Producer-Consumer Pattern


\hd{Architecture}

3-tier client-server architecture

DSL (domain-specific language)

microservices


\hd{Real Life}

You're paid for working code, not diagrams and design documentation

The users don't sign the checks

``The code you write today pays for your time to debug it tomorrow.''
\end{col3}

\pagebreak
\subsection*{Grading}

% TODO percentages
Each part of the course will be weighted in your final grade for the course as
follows:

\begin{adjustwidth}{3em}{}
\begin{NiceTabular}{lr}
Participation & 50\% \\
Learning notes & 5\% \\
Presentations & 25\% \\
Final exam & 20\% \\
\end{NiceTabular}
\end{adjustwidth}

To get a passing grade, however (a grade of C– or better), you must score at
least 50\% in each of the above categories.

Letter-grade cut-offs are as follows, where $s$ is your overall score for the
course:

\vspace{1em}
\begin{adjustwidth}{3em}{}
\begin{NiceTabular}{cl}
$93 \le s$ & A \\
$90 \le s < 93$ & A– \\
$87 \le s < 90$ & B+ \\
$84 \le s < 87$ & B \\
$80 \le s < 84$ & B– \\
$77 \le s < 80$ & C+ \\
$74 \le s < 77$ & C \\
$70 \le s < 74$ & C– \\
$65 \le s < 70$ & D+ \\
$60 \le s < 65$ & D \\
$s < 60$ & F
\end{NiceTabular}
\end{adjustwidth}

\subsection*{Additional Information}

\begin{itemize}
\item Students are responsible for knowing the information about campus
policies, procedures, and resources on the Syllabus Addendum website linked
below. The site includes topics such as learning outcomes; registration
policies; academic honesty policy; attendance and disruptive behavior policy;
standards for student conduct; prevention and reporting of discrimination,
harassment, and retaliation; animals on campus policy; emergency procedures;
resources for students with disabilities; learning and advising resources;
counseling and psychological services; financial aid; IT Help; and more.
\url{https://academicprograms.humboldt.edu/content/syllabus-addendum}

\item It is the student's responsibility to notify the instructor in advance
of the need for accommodations and to provide documentation to the university
(SDRC or Dean of Students).

\item If you do any online communication in connection with this course,
remember that university regulations regarding disruptive behavior extend to
the online environment. Appropriate online behavior (i.e., netiquette) is
expected.
\end{itemize}

This syllabus is subject to change. Changes will be announced on Canvas. Be
sure to check your Cal Poly Humboldt email regularly so you're aware of course
updates and announcements.

\end{document}
